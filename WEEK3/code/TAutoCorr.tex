\documentclass[12pt]{article}

\title{Autocorrelation in weather}

\author{Yige Sun}

\usepackage{Sweave}
\begin{document}
  \maketitle
\input{TAutoCorr-concordance}

\section*{1.Calculate the correlation coefficient}
  First, I loaded the file, shifted the tempperature data and cut them to same length, and worked out the correlation coefficient with following codes.
\begin{Schunk}
\begin{Sinput}
> load("/home/yige/Documents/CMEECoursework/WEEK3/data/KeyWestAnnualMeanTemperature.RData")
> y1901<-ats$Temp[-100]
> y1902<-ats$Temp[-1]
> corcoe<-cor(y1901,y1902)
\end{Sinput}
\end{Schunk}
  
  The temperature flactuation over years is shown in the plot below

\begin{Schunk}
\begin{Sinput}
> plot(Temp~Year, data = ats, type="l",ylab="Temperature")
\end{Sinput}
\end{Schunk}
\includegraphics{TAutoCorr-002}
  
  And the correlation looks like;
\begin{Schunk}
\begin{Sinput}
> plot(y1902~y1901, pch=16)
\end{Sinput}
\end{Schunk}
\includegraphics{TAutoCorr-003}

\section*{2.Randomize the data, and repeat the calculation}
  Here, I created a empty list, and randomize the data, generate correlation coefficients for 10000 times and store them to the list.
\begin{Schunk}
\begin{Sinput}
> rcorcoe <- c()
> for (i in 1:10000){
+   rtemp<- sample(ats$Temp,size = length(ats$Temp),replace = FALSE)
+   ry1901<-rtemp[-100]
+   ry1902<-rtemp[-1]
+   rcorcoe[i]<- cor(ry1901,ry1902)
+ }
\end{Sinput}
\end{Schunk}
The density of randomly generated correlation coefficient are:

\begin{Schunk}
\begin{Sinput}
> plot(density(rcorcoe),main = "Density of randomly generated correlation coefficient")
\end{Sinput}
\end{Schunk}
\includegraphics{TAutoCorr-005}
\section*{3.Calculate p value}
  Finally, I compared the each random correlation coefficient with the actual one which a loop, and work out how many of them are greater that the actual one with following codes:
\begin{Schunk}
\begin{Sinput}
> for (i in rcorcoe){
+   a <- 0
+   if(i>corcoe){a = a+1}
+   p=a/length(rcorcoe)
+ }
> print(paste("p-value=",p))
\end{Sinput}
\begin{Soutput}
[1] "p-value= 0"
\end{Soutput}
\end{Schunk}
The p value is 0
\section*{Conclusion}
The p value is 0 means none of the randomly generated correlation coefficient is greater than the actual one, and temperature growth is not randomly occured, there is a significant correlation between current and previous year.
\end{document}
