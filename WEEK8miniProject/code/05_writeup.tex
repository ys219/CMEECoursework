%initial set up
\documentclass[11pt, a4paper]{article}
% page layout
\usepackage{geometry}
\geometry{left=2cm,right=2cm,top=2cm,bottom=2cm}
% page style anf spaceing setting
\pagestyle{plain}
\linespread{2}

%\setlength{\parindent}{0cm}
%\usepackage{helvet}
%\renewcommand{\familydefault}{\sfdefault}
\usepackage{graphicx}
\usepackage{lineno}
\usepackage[numbers]{natbib}

\begin{document}
 
	
	\begin{titlepage} 
		\centering 
		
		
		\vspace*{\baselineskip} 
		
		%------------------------------------------------
		%	Title
		%------------------------------------------------
		
		\vspace{0.75\baselineskip} % Whitespace above the title
		
		{\LARGE \textbf{Phenomenological vs Mechanistic models: what are the best fits for empirical Functional Response data}} % Title
		
		\vspace{0.1\baselineskip} % Whitespace below the title
		
		\rule{\textwidth}{1.6pt} % Thick horizontal rule
		
		\vspace{1\baselineskip} % Whitespace after the title block
		MRes. Computational Methods in Ecology and Evolution
		
		{\large \textbf{Miniproject}}\\ %large font
		%------------------------------------------------
		%	name
		%------------------------------------------------
		\vspace*{2\baselineskip} % Whitespace under the course
		{\Large Yige Sun} \\% large font	
		CID: 01768241\\
		yige.sun19@imperial.ac.uk
		
		\vfill % Whitespace between editor names and publisher logo
		%------------------------------------------------
		%	year
		%------------------------------------------------
		
		{\LARGE 2019-2020} 
		\vspace{1\baselineskip}
	\end{titlepage}

\linenumbers
\section{Abstract}
not decided ye will write finally

\section{Introduction}
In nature, the food consumption rate of would respond to the changing density of accessible resources. This feeding relationship was termed as the Functional response. In 1959, the pioneer in this field Holling suggested the model of functional response arise from an experiment of the small mammal's predation behaviour of the European Pine Saw fly \cite{Holling1959}. It addressed different influential components of predation and parametrized them to form mechanistic models. Since then,  

The functional responses were classified into three types: Type I described a linear relationship between resource density and consumption rate. It is neglecting the handling process in the entire progress. Only one parameter, the capture rate(\textit{a}), was considered in this model. It is constant across the entire predation behaviour. As the type I response has only one parameter and linearly described the relationships, it is commonly applied on the explaination of passive feeder \cite{Jeschke2004}, such as spider which capture prey with web, and oyster, a sessile filter feeder that solely rely on the resource density to increase the  consumption rate. 

\bibliographystyle{unstrnat}
\bibliography{05_references}
\end{document}